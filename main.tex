\documentclass[12pt]{article}
\usepackage{geometry}
\geometry{
	a4paper,
	lmargin=2.5cm, %Seitenrand left
	tmargin=2.5cm, %Seitenrad top
	headsep=35pt %Abstand von Kopfzeile
}
\begin{document}
	
	\tableofcontents
	\addtocontents{toc}{\vspace{0.65cm}}
	\newpage
	\thispagestyle{plain}
	
	
	\section{Konzept für das definierte Vorgehensmodell und dazugehörige Beschreibung der Liefergegenstände und Meilensteine}
	
	Um eine effektive und strukturierte Umsetzung des Projekts sicherzustellen, wird ein definiertes Vorgehensmodell verwendet. Das Vorgehensmodell basiert auf bewährten Methoden und ermöglicht eine klare Planung, Durchführung und Kontrolle des Entwicklungsprozesses. Die folgenden Elemente sind Teil des Vorgehensmodells:
	
	\subsection{Projektinitialisierung}
	- Kick-off-Meeting mit der BBQ GmbH: In diesem Meeting werden die Projektziele, Anforderungen und Erwartungen gemeinsam mit der BBQ GmbH besprochen und festgelegt.
	- Definition der Projektorganisation: Es wird ein Projektteam zusammengestellt, das aus Entwicklern, Testern, Projektmanagern und anderen relevanten Stakeholdern besteht.
	- Erstellung des Projektplanes: Der Projektplan umfasst die Ressourcenplanung, Zeitplanung, Budgetplanung und die Definition der Meilensteine.
	
	\subsection{Anforderungsanalyse}
	- Anforderungserhebung: Eine detaillierte Analyse der Anforderungen wird durchgeführt, um alle Funktionalitäten und Schnittstellen der Software zu erfassen. Dies erfolgt in enger Zusammenarbeit mit der BBQ GmbH, um sicherzustellen, dass alle spezifischen Anforderungen berücksichtigt werden.
	- Pflichtenheft oder Product Backlog: Die erhobenen Anforderungen werden in einem Pflichtenheft oder Product Backlog dokumentiert. Dabei werden die Anforderungen priorisiert und ihre Umsetzung beschrieben.
	
	\subsection{Design und Entwicklung}
	- Erstellung des Software-Designs: Basierend auf den Anforderungen wird ein umfassendes Software-Design entwickelt, das die Architektur, Datenstrukturen und Benutzeroberfläche beschreibt.
	- Iterative Entwicklung: Die Software wird in iterativen Schritten entwickelt, wobei in jeder Iteration bestimmte Funktionalitäten implementiert und getestet werden.
	- Code-Reviews und Qualitätssicherung: Regelmäßige Code-Reviews und Tests werden durchgeführt, um die Qualität des entwickelten Codes sicherzustellen.
	
	\subsection{Testing und Qualitätssicherung}
	- Testplanung: Ein detaillierter Testplan wird erstellt, der alle erforderlichen Tests und Testfälle umfasst, um die korrekte Funktionalität und Stabilität der Software zu gewährleisten.
	- Durchführung von Tests: Die entwickelte Software wird umfassend getestet, einschließlich Modul-, Integrations- und Systemtests. Dabei werden sowohl automatisierte als auch manuelle Tests durchgeführt.
	- Fehlermanagement: Gefundene Fehler werden erfasst, dokumentiert und behoben. Ein Bug-Tracking-System wird verwendet, um einen klaren Überblick über den Fortschritt bei der Fehlerbehebung zu haben.
	
	\subsection{Abnahme und Auslieferung}
	- Abnahme der Software: Die BBQ GmbH nimmt die entwickelte Software ab und überprüft, ob sie alle definierten Anforderungen erfüllt.
	- Auslieferung der Software: Nach der Abnahme wird die Software an die BBQ GmbH ausgeliefert und in Betrieb genommen
	
	\section{Konzept zur Sicherstellung der Qualität des Projektergebnisses}
	Um sicherzustellen, dass das Projektergebnis der Software zur Zeiterfassung von Mitarbeitern höchste Qualität aufweist, wird ein umfassendes Konzept zur Qualitätssicherung implementiert. Dieses Konzept umfasst mehrere Schritte und Maßnahmen, um sicherzustellen, dass die Software den Anforderungen der BBQ GmbH entspricht und ein reibungsloser Betrieb gewährleistet ist. Die folgenden Punkte sind Teil des Konzepts:
	
	\subsection{Definition von Qualitätsstandards}
	 Es werden klare Qualitätsstandards definiert, anhand derer die Softwareentwicklung und das Endprodukt bewertet werden können. Diese Standards werden in Zusammenarbeit mit der BBQ GmbH festgelegt und berücksichtigen branchenspezifische Anforderungen sowie die Einhaltung gesetzlicher Vorschriften, insbesondere des Arbeitszeitgesetzes (ArbZG).
	
	\subsection{Umfassendes Testing}
	Die Software wird einer gründlichen Testphase unterzogen, um sicherzustellen, dass alle Funktionen einwandfrei funktionieren und den definierten Anforderungen entsprechen. Dies umfasst sowohl automatisierte Tests als auch manuelle Tests durch erfahrene Tester. Es werden verschiedene Testarten wie Modul-, Integrations- und Systemtests durchgeführt, um eine umfassende Überprüfung der Software sicherzustellen.
	
	\subsection{Fehlermanagement und Bug-Tracking}
	Es wird ein effektives Fehlermanagement implementiert, um Fehler und Probleme während des Entwicklungsprozesses zu identifizieren, zu dokumentieren und zu beheben. Ein Bug-Tracking-System wird eingerichtet, um einen klaren Überblick über alle gefundenen Fehler zu haben und sicherzustellen, dass sie zeitnah behoben werden.
	
	\subsection{Code-Qualitätssicherung}
	Es werden bewährte Methoden und Standards für die Entwicklung von qualitativ hochwertigem Code angewendet. Dies umfasst eine einheitliche und konsistente Codeformatierung, die Verwendung von Kommentaren für eine bessere Lesbarkeit, regelmäßige Code-Reviews und die Vermeidung von bekannten Anti-Patterns und Performance-Problemen.
	
	\subsection{Usability-Testing}
	Die Benutzerfreundlichkeit der Software wird sorgfältig überprüft, um sicherzustellen, dass die Benutzeroberfläche intuitiv und einfach zu bedienen ist. Usability-Tests mit echten Benutzern werden durchgeführt, um Feedback zu sammeln und Verbesserungen vorzunehmen.
	
	\subsection{Sicherheitsprüfungen}
	Die Sicherheit der Software steht im Fokus. Es werden umfangreiche Sicherheitsprüfungen durchgeführt, um mögliche Schwachstellen zu identifizieren und zu beheben. Dies umfasst die Absicherung von Schnittstellen, den Schutz sensibler Daten und die Implementierung von Zugriffskontrollen und Verschlüsselungstechnologien.
	
	\subsection{Dokumentation}
	Eine umfassende Dokumentation wird erstellt, die die Funktionalitäten, Schnittstellen, Konfigurationen und Verwendung der Software detailliert beschreibt. Dies umfasst auch Benutzerhandbücher und Schulungsmaterialien, um den Mitarbeitern der BBQ GmbH die effektive Nutzung der Software zu ermöglichen
 	
 	
 	
 	
 	
 	
	
\end{document}